\section{Introduction}
\label{sec:introduction}
Quantifier/ variable elimination is a fundamental problem in elementary real arithmetic Virtual substitution method provides an incomplete solution to the problem. In 1993, the concept of virtual substitution was first introduced \cite{weispfenning}. Virtual substitution was already used earlier to quantifier elimination for linear real arithmetic formulas. Now, virtual substitution has extended to quantifier elimination for non-linear real arithmetic formulas. But, this method is restricted
to formulas that are linear or quadratic in the quantified variable. It means virtual substitution method is applicable if quantified variables has a degree of at most two. In addition, applying the method iteratively to eliminate quantified variables may increase the degrees of remaining variables, thus violating the degree restrictions.\newline
Section $2$ consists preliminaries with some definitions. How to construct the real zeros and virtual substitution rules by which we can eliminate quantified variables from non-linear real arithmetic formulas are explained in Section $3$. In the Section $4$ we conclude the paper.
