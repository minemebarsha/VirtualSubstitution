\section{Quantifier Elimination with the Virtual Substitution}
\label{sec:quantifier-elimination-with-the-virtual-substitution}
Let $\varphi^\mathbb{R}$ is a quantifier-free real-arithmetic formula where $x\in \varphi^\mathbb{R}$ and $x$ has at most degree $2$ in $\varphi^\mathbb{R}$. After eliminating existential qualifier with virtual substitution we will get the followings,
$$ \exists x. \varphi^\mathbb{R} \Longleftrightarrow \bigvee\limits_{t\in T(x,\varphi^\mathbb{R})}  (\varphi^\mathbb{R} [t\backslash\backslash x] \wedge C_t)$$

Again, let us consider $\exists x.\varphi$ where, $\varphi = (x^{2}y + x + y = 0) \wedge (y^{2} -2 < 0)$. We already constructed all the test candidates for $x$ and $y$. Also we get to know about substitution rules with virtual substitution. By using the substitution rules, we eliminate all occurrences of $x$ and $y$ from $\varphi$. Then we get the following equivalence holds.
$$ \exists x\exists y. \varphi \Longleftrightarrow \bigvee\limits_{t_{1}\in T(x,\varphi)}(\bigvee\limits_{t\in T(y,\varphi^{\prime})}  (\varphi^{\prime} [t_{2}\backslash\backslash y] \wedge C_{t_{2}}))$$
where, $\varphi^{\prime} = (\varphi [t_{1}\backslash\backslash x] \wedge C_{t_{1}})$.

The solutions of $x$ and $y$ foe $\varphi$ is shown in the figure 2. Also we can briefly understand the whole procedure of virtual substitution though figure 3.

\input{ro_4.tex}
\input{ro_1.tex}
