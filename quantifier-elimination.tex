\section{Quantifier Elimination with the Virtual Substitution}
\label{sec:quantifier-elimination-with-the-virtual-substitution}
Let $\varphi^\mathbb{R}$ is a quantifier-free real-arithmetic formula where $x\in \varphi^\mathbb{R}$ and $x$ has at most degree $2$ in $\varphi^\mathbb{R}$. After eliminating existential qualifier with virtual substitution we will get the followings,
$$ \exists x. \varphi^\mathbb{R} \Longleftrightarrow \bigvee\limits_{t\in T(x,\varphi^\mathbb{R})}  (\varphi^\mathbb{R} [t\backslash\backslash x] \wedge C_t)$$

Again, let us consider $\exists x.\varphi$ where, $\varphi = (x^{2}y + x + y = 0) \wedge (y^{2} -2 < 0)$. We already constructed all the test candidates for $x$ and $y$. Also we get to know about substitution rules with virtual substitution. By using the substitution rules, we eliminate all occurrences of $x$ and $y$ from $\varphi$. Then we get the following equivalence holds.
$$ \exists x\exists y. \varphi \Longleftrightarrow \bigvee\limits_{t_{1}\in T(x,\varphi)}(\bigvee\limits_{t\in T(y,\varphi^{\prime})}  (\varphi^{\prime} [t_{2}\backslash\backslash y] \wedge C_{t_{2}}))$$
where, $\varphi^{\prime} = (\varphi [t_{1}\backslash\backslash x] \wedge C_{t_{1}})$.

The solutions of $x$ and $y$ foe $\varphi$ is shown in the figure 2. Also we can briefly understand the whole procedure of virtual substitution though figure 3.

\begin{figure}[htb]
\definecolor{ffvvqq}{rgb}{0.44313725490196076,0.44313725490196076,0.44313725490196076}
\definecolor{qqqqff}{rgb}{0.3333333333333333,0.3333333333333333,0.3333333333333333}
\begin{tikzpicture}
	[line cap=round,line join=round,>=triangle 45,x=2.2239498017171777cm,y=1.4852526944330273cm]
	\draw[->,color=black] (-3.1381482009239074,0.) -- (3.6066083526718855,0.);
	\foreach \x in {-3.,-2.5,-2.,-1.5,-1.,-0.5,0.5,1.,1.5,2.,2.5,3.,3.5}
	\draw[shift={(\x,0)},color=black] (0pt,2pt) -- (0pt,-2pt);
	\draw[->,color=black] (0.,-2.134493672850494) -- (0.,1.2319368468454657);
	\foreach \y in {-2.,-1.5,-1.,-0.5,0.5,1.}
	\draw[shift={(0,\y)},color=black] (2pt,0pt) -- (-2pt,0pt);
	\clip(-3.1381482009239074,-2.134493672850494) rectangle (3.6066083526718855,1.2319368468454657);
	\draw[line width=1.2pt,color=qqqqff,smooth,samples=100,domain=-3.1381482009239074:3.6066083526718855] plot(\x,{(\x)^(2.0)-2.0});
	\draw[line width=1.2pt,color=ffvvqq,smooth,samples=100,domain=-3.1381482009239074:3.6066083526718855] plot(\x,{1.0-4.0*(\x)^(2.0)});
	\begin{scriptsize}
	\draw[color=qqqqff] (-1.4965672407894814,0.7888289427512096) node {$y^{2}-2$};
	\draw[color=ffvvqq] (-0.9731646158190848,-1.6854380116543015) node {$1-4y^{2}$};
	\draw [color=black] (-0.5,0.)-- ++(-2.5pt,-2.5pt) -- ++(5.0pt,5.0pt) ++(-5.0pt,0) -- ++(5.0pt,-5.0pt);
	\draw[color=black] (-0.6222469468048416,0.1881054754556408) node {$\frac{-1}{2}$};
	\draw [fill=black] (-1.4142135623730951,0.) circle (1.5pt);
	\draw[color=black] (-1.526306026299163,-0.13307340804892073) node {$-\sqrt{2}$};
	\draw [color=black] (0.5,0.)-- ++(-2.5pt,-2.5pt) -- ++(5.0pt,5.0pt) ++(-5.0pt,0) -- ++(5.0pt,-5.0pt);
	\draw[color=black] (0.6029910161940414,0.19405323255757712) node {$\frac{1}{2}$};
	\draw [fill=black] (1.4142135623730951,0.) circle (1.5pt);
	\draw[color=black] (1.3167218684264004,0.11673239023240492) node {$\sqrt{2}$};
	\draw [fill=black] (-1.3121867706294554,0.) circle (1.5pt);
	\draw[color=black] (-1.1397018146733018,0.15241893284402286) node {-$\surd$2+$\varepsilon$};
	\draw [color=black] (0.06769287701977207,0.)-- ++(-2.5pt,-2.5pt) -- ++(5.0pt,5.0pt) ++(-5.0pt,0) -- ++(5.0pt,-5.0pt);
	\draw[color=black] (0.15690923354881695,0.15241893284402286) node {$0 + \epsilon$};
	\draw [fill=black] (1.5,0.) circle (1.5pt);
	\draw[color=black] (1.6319529948290257,-0.1271256509469844) node {$\sqrt{2} + \epsilon$};
	\end{scriptsize}
\end{tikzpicture}
\caption{Substitution of Infinitesimal Expressions in $y^{2}-2<0$}
\label{fig:graph}
\end{figure}
\begin{figure}[htb]
	\definecolor{uuuuuu}{rgb}{0.26666666666666666,0.26666666666666666,0.26666666666666666}
	\begin{tikzpicture}[line cap=round,line join=round,>=triangle 45,x=1.4506228748477659cm,y=0.9554504827579918cm]
	\draw[->,color=black] (-4.541195720862315,0.) -- (5.799189957651242,0.);
	\foreach \x in {-4.5,-4.,-3.5,-3.,-2.5,-2.,-1.5,-1.,-0.5,0.5,1.,1.5,2.,2.5,3.,3.5,4.,4.5,5.,5.5}
	\draw[shift={(\x,0)},color=black] (0pt,2pt) -- (0pt,-2pt);
	\draw[->,color=black] (0.,-3.3209773598931354) -- (0.,1.9121562145094437);
	\foreach \y in {-3.,-2.5,-2.,-1.5,-1.,-0.5,0.5,1.,1.5}
	\draw[shift={(0,\y)},color=black] (2pt,0pt) -- (-2pt,0pt);
	\clip(-4.541195720862315,-3.3209773598931354) rectangle (5.799189957651242,1.9121562145094437);
	\draw[line width=1.2pt,smooth,samples=100,domain=-4.541195720862315:5.799189957651242] plot(\x,{(\x)^(3.0)+2.0*(\x)});
	\draw[line width=1.2pt,smooth,samples=100,domain=-4.541195720862315:5.799189957651242] plot(\x,{(\x)^(2.0)-2.0});
	\draw [line width=1.2pt] (0.,-3.3209773598931354) -- (0.,1.9121562145094437);
	\begin{scriptsize}
	\draw[color=black] (-1.4390800173082487,-2.902866172247569) node {y3 + 2y};
	\draw[color=black] (-2.2213526729871003,1.3321955348720445) node {y2 - 2};
	\draw[color=black] (-0.3780665302955535,1.4400951961999326) node {y = 0};
	\draw [fill=uuuuuu] (-1.4142135623730951,0.) circle (1.5pt);
	\draw[color=uuuuuu] (-1.6728626500398596,-0.18739136216238161) node {-root(2)};
	\draw [fill=uuuuuu] (1.4142135623730951,0.) circle (1.5pt);
	\draw[color=uuuuuu] (1.2494202591052757,0.3251320291450875) node {root(2)};
	\draw [color=black] (0.,0.)-- ++(-2.5pt,-2.5pt) -- ++(5.0pt,5.0pt) ++(-5.0pt,0) -- ++(5.0pt,-5.0pt);
	\draw [fill=black] (-1.2952137817811031,0.) circle (1.5pt);
	\draw[color=black] (-0.8636150751996683,0.3341236675890782) node {-root(2) +epsilon};
	\draw [fill=black] (1.5,0.) circle (1.5pt);
	\draw[color=black] (1.8518601203751959,-0.18739136216238161) node {root(2) + ep};
	\end{scriptsize}
	\end{tikzpicture}
\caption{Substitution of Infinitesimal Expressions in $y^{2}-2<0$}
\label{fig:graph}
\end{figure}
