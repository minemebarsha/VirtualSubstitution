\section{Solving Non-linear Equalities with Virtual Substitution}
\label{sec:solving non-linear equalities with virtual substitution}
VS is a restricted but very efficient procedure to solve non-linear equalities. In the paper [1] author explored an extension of the ideas in [2] from the linear to the quadratic case. For linear case the idea was to eliminate a quantifier from $\exists x \varphi$ by replacing $x$ in $\varphi$ with $t$ that may involve improper expressions such as $\pm \infty$ or $\epsilon$. However, $\varphi [t\backslash\backslash x]$ is defined in such a way that these improper expressions do not occur in the resulting formula.

Author extended these idea to various quadratic cases. The cases are the substitution of SREs and the substitution of infinitesimal expressions in formulas.

In virtual substitution, first a variable is replaced by test candidate and to perform the replacement we need to construct test candidates. An univariate real-arithmetic formula is satisfiable if and only if there is one test candidate for which satisfies formula and the side conditions of t holds. For multivariate real-arithmetic formula the virtual substitution method continues with the elimination of the next variable.

In this section we will see how we can apply virtual substitution. Let us consider a multivariate real-arithmetic formula which we will use in this section,
$$ \varphi^\mathbb{R} = \underbrace{(x^{2}y + x + y = 0)}\limits_{p_{1}} \wedge \underbrace{(y^{2} -2 < 0)}\limits_{p_{2}}$$
\subsection{Constructing test candidates with side condition}
First we will eliminate $x$ from $\varphi^\mathbb{R}$. To construct the test candidates for x, we have to compute SqrtEx for $x$. Also we need to consider an infinitesimal $\epsilon$.
\begin{mdframed}[hidealllines=true,backgroundcolor=blue!20,innerleftmargin=3pt,innerrightmargin=3pt,leftmargin=-3pt,rightmargin=-3pt]
	\begin{definition}[Construction of Test Candidates]
		\label{def:construction_of_test_candidates}
		The set of all test candidates is defined by,
		$$ TCS := SqrtEx \cup \{t+\epsilon \text{ } \lvert  \text{ }t\in SqrtEx\} $$
		The set of test candidates for $x$ in $p(x) = ax^{2} + bx + c \sim 0$ is defined by,
		$$
		(x, p(x)\sim 0)
		\quad \mapsto \quad 
		\left\{
		\begin{array}{ll}
		{\displaystyle \{-\infty,\frac{-c}{b},\frac{-b \pm \sqrt{b^{2}-4ac}}{2a}\}} 
		& 
		\text{, if }\sim \text{ is weak }
		\\[0.6cm] % adjust the line spacing
		{\displaystyle \{-\infty,\frac{-c}{b} + \epsilon,\frac{-b \pm \sqrt{b^{2}-4ac}}{2a}+\epsilon\}}
		& 
		\text{, otherwise }
		\end{array}
		\right.$$
		where $a, b, c \in P$, $x\notin a\cup b\cup c $ and weak means $\{=,\leq,\geq\}$
		
		The side condition of a test candidate is defined by,
		$$
		C_t: t
		\quad \mapsto \quad 
		\left\{
		\begin{array}{lll}
		{\displaystyle C_{t^{\prime}}}
		& 
		\text{, if } t = t^{\prime} + \epsilon
		\\[0.6cm] % adjust the line spacing
		{\displaystyle \{s\neq 0 \wedge r\geq 0\}}
		& 
		\text{, if } t = \frac{p+q\sqrt{r}}{s}
		\\[0.6cm] % adjust the line spacing
		{\displaystyle true}
		& 
		\text{, Otherwise }
		\end{array}
		\right.$$
		where $p, q, r$ and $s$ are polynomials and $t^{\prime}$ is a test candidate where $\epsilon \notin t^{\prime}$.
	\end{definition}
\end{mdframed}
By eliminating $x$ first we get the following test candidates,
\begin{alignat}{2}
	&x_{0} = -y\qquad                            
	&& \text{, if $ y = 0 \wedge 1\neq 0$} \\
	&x_{1} = \frac{-1 + \sqrt{1^{2}-4y^{2}}}{2y}\qquad      
	&& \text{, if $ y \neq 0 \wedge 1^{2}-4y^{2}\geq 0$} \\
	&x_{2} = \frac{-1 - \sqrt{1^{2}-4y^{2}}}{2y}\qquad      
	&& \text{, if $ y \neq 0 \wedge 1^{2}-4y^{2}\geq 0$} \\
	&x_{3} = -\infty\qquad      
	&& \text{, if $ y = 0 \wedge 1 = 0$}
\end{alignat}
Here, $C_{x_{3}}$ is invalid. So, $x_{3}$ can not be a real zero of $p_{1}$